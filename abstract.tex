\begin{abstract}

The underrepresentation in computer science of women, domestic students of color, and students of lower socioeconomic status remains a national issue. Recent studies demonstrate two critical factors: Persistent stereotypes about ``who does computer science'' can preclude interest in the field for members of these groups; many also perceive computing as ``irrelevant'' and ``asocial''.  While these issues must be addressed at multiple ages and levels, many suggest starting early, before students have developed stereotypes.

As a step in combating (mis-)perceptions of ability and relevance, we designed and conducted a spectrum of week-long summer ``code camps'' for regional middle-school students. These camps emphasize meaningful uses of computing, building self-efficacy, and broadening understanding of who does and can do computer science.  In this paper, we focus on our ``data science for social good'' (ds4sg) camp, in which students explored computational approaches to data science through a lens of computing for social good, discovering how computing helps them not only better understand societal issues but also convince others to address problems.

We discuss the rationale for the curriculum and its content, including our uses of pair programming, personal projects, and a growth model that brings students from block-based programming to professional Jupyter data notebooks. We consider the short-term effects the camps have on students' self-efficacy and perceptions of computer science. We conclude with recommendations and guidelines for those intending to offer similar camps.

\end{abstract}
